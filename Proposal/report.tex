\documentclass[10pt]{article}
\usepackage{graphicx}% Include figure files
\usepackage{dcolumn}% Align table columns on decimal point
\usepackage{bm}% bold math
\usepackage{verbatim}
%\usepackage[mathlines]{lineno}% Enable numbering of text and display math
%\linenumbers\relax % Commence numbering lines

\usepackage[utf8]{inputenc}
\usepackage[T1]{fontenc}
\usepackage{mathptmx}
\usepackage{etoolbox}

\begin{document}

\title{Title}
\author{Myeong-gi Jo, Jeongmin Lee, Subin Kim and Eugene Park}

\maketitle

\section{Introduction}
Background : What is negative resistance?
How was and is it used?
- Chua circuit
- 
How it is linked to module 2 and 3 (Op-amp, mosfets are the main of module 2 and 3)
\\
Different methods on creating negatived differential resistance in circuits and research using their applications.
- op-amp -> Chua circuit
- 
\section{Research Design and Methodology}
Research design - enable inverted op amp
add mosfets the above for increased nonlinearity

anlayze IV curve characteristics and Impedance if possible

\section{Preliminary Implications}
How this device is expected to be different in application:
- Difference from the negative differential resistance shown in the above
- Novel design
- How the added nonlinearities can present diverse phenomena.
\section{Conclusion}


\end{document}
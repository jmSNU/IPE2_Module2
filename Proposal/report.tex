\documentclass[12pt]{article}
\usepackage{graphicx}% Include figure files
\usepackage{dcolumn}% Align table columns on decimal point
\usepackage{bm}% bold math
\usepackage{verbatim}
%\usepackage[mathlines]{lineno}% Enable numbering of text and display math
%\linenumbers\relax % Commence numbering lines

\usepackage[utf8]{inputenc}
\usepackage[T1]{fontenc}
\usepackage{mathptmx}
\usepackage{etoolbox}
\usepackage[utf8]{inputenc}
\usepackage{kotex}
\usepackage{graphicx}
\usepackage{subfigure}
\usepackage{titling}
\setlength{\droptitle}{-2cm}
\usepackage[utf8]{inputenc}
\usepackage{array}
\usepackage[utf8]{inputenc}
\usepackage{kotex}
\usepackage{graphicx}
\usepackage{subfigure}
\usepackage{titling}
\setlength{\droptitle}{-2cm}
\usepackage[utf8]{inputenc}
\usepackage{array}
\usepackage{amssymb}
\usepackage{amsmath}
\usepackage{siunitx} 
\usepackage{enumerate} 
\usepackage{pgfplots}
\usepackage{pgfplotstable}
\usepackage{tikz,pgfplots}
\usepackage{amsmath}  
\usepackage{wasysym}
\usepackage{geometry}
\usepackage{authblk}
\usepackage{kotex}
\usepackage{bibunits}
\usepackage{tabularx}
\usepackage{hyperref}
\usepackage{amssymb}
\usepackage{amsmath}
\usepackage{siunitx} 
\usepackage{enumerate} 
\usepackage{pgfplots}
\usepackage{pgfplotstable}
\usepackage{tikz,pgfplots}
\usepackage{amsmath}  
\usepackage{wasysym}
\usepackage{geometry}
\usepackage{authblk}
\usepackage{kotex}
\usepackage{bibunits}
\usepackage{tabularx}
\usepackage{hyperref}
\usepackage{listings}
\geometry{
    a4paper,
    total={170mm,257mm},
    left=20mm,
    top=20mm,
}

\begin{document}

\title{Title}
\author{Myeong-gi Jo, Jeong-min Lee, Subin Kim and Eugene Park}

\maketitle

\section{Introduction}
\noindent\fbox{%
    \parbox{\textwidth}{%
        \textbf{Outline}\\
        Background : What is negative resistance?\\
        How was and is it used?\\
        - Chua circuit\\
        How it is linked to module 2 and 3 (Op-amp, mosfets are the main of module 2 and 3)
        \\
        Different methods on creating negatived differential resistance in circuits and research using their applications.\\
        - op-amp -> Chua circuit\\

        }%
}



In the realm of electrical circuits, the relationship between voltage and current is typically described as linear for ohmic devices, resulting in a straight line on the current-voltage (\(I-V\)) plane. However, for non-linear devices, this \(I-V\) curve takes on a more complex shape, deviating from linearity. Within this domain of non-linear components, there emerges a fascinating phenomenon known as negative resistance. Unlike ordinary resistors, which consistently exhibit positive resistance, negative resistance is characterized by certain intervals or regions in which an increase in voltage coincides with a decrease in current, causing the \(I-V\) curve to display this intriguing property.

One of the most renowned circuits that effectively leverages negative resistance is the Chua circuit, where classical chaotic behavior manifests. Central to this circuit's dynamics is the Chua diode, which serves as a negative active resistor purposely designed to produce positive feedback. The simplicity of the Chua circuit facilitated extensive exploration of chaotic system characteristics by physicists and engineers during the 1990s\cite{impact_chua}.

The design of non-linear circuits, and by extension, the incorporation of negative resistance, plays a pivotal role in the pursuits of physicists and engineers. One notable approach involves creating Chua diodes by combining operational amplifiers (OP-AMPs) with several linear components, resulting in a distinctive three-segment piecewise curve on the current-voltage ($I-V$) characteristic \cite{chua_circuit}. Additionally, Negative Impedance Converters (NIC) utilize negative resistors, which are implemented using metal-oxide-semiconductor field-effect transistors (MOSFETs) or bipolar junction transistors (BJTs). The concept of NIC dates back to LINVILL's groundbreaking work in 1953, where he introduced the use of BJTs for this purpose. More recently, innovative circuits based on MOSFETs have been reported, particularly in the 2010s.\cite{NIC_BJT,NIC_MOSFET,NIC_MOSFET2}

The unique characteristics of negative resistance necessitate the use of non-linear devices, such as operational amplifiers (OP-AMPs) and transistors (BJT or MOSFET). To understand and quantify the non-linear behavior inherent to these devices, the application of metrological techniques becomes indispensable. The importance of metrology becomes evident when examining complex systems, particularly chaotic ones. Chaos, by definition, is exquisitely sensitive to initial conditions and is amenable to mathematical or numerical control. However, the experimental realization of a chaotic system is a challenging task due to the myriad of uncontrollable factors at play. Consequently, comprehending potential influences on the circuit and obtaining precise parameters are paramount to the success of such experiments. In this context, metrology emerges as an essential component in the design of negative resistors.




\section{Research Design and Methodology}

\noindent\fbox{%
    \parbox{\textwidth}{%
    \textbf{Outline}\\
        Research design - enable inverted op amp\\
        add mosfets the above for increased nonlinearity\\
        anlayze IV curve characteristics and Impedance if possible    }%
}


\section{Preliminary Implications}

\noindent\fbox{%
    \parbox{\textwidth}{%
    \textbf{Outline}\\
    How this device is expected to be different in application:\\
    - Difference from the negative differential resistance shown in the above\\
    - Novel design\\
    - How the added nonlinearities can present diverse phenomena.
        }%
}

\section{Conclusion}

\bibliographystyle{unsrt}
\bibliography{../Proposal/references.bib}

\end{document}
\documentclass[%
 aip,
amsmath,amssymb,
reprint,
]{revtex4-1}

\usepackage{graphicx}% Include figure files
\usepackage{dcolumn}% Align table columns on decimal point
\usepackage{bm}% bold math
\usepackage{verbatim}
%\usepackage[mathlines]{lineno}% Enable numbering of text and display math
%\linenumbers\relax % Commence numbering lines

\usepackage[utf8]{inputenc}
\usepackage[T1]{fontenc}
\usepackage{mathptmx}
\usepackage{etoolbox}

%% Apr 2021: AIP requests that the corresponding 
%% email to be moved after the affiliations
\makeatletter
\def\@email#1#2{%
 \endgroup
 \patchcmd{\titleblock@produce}
  {\frontmatter@RRAPformat}
  {\frontmatter@RRAPformat{\produce@RRAP{*#1\href{mailto:#2}{#2}}}\frontmatter@RRAPformat}
  {}{}
}%
\makeatother
\begin{document}

\preprint{AIP/123-QED}

\title[Intermediate Physics Laboratory 2, Module 2]{title}
% Force line breaks with \\
\author{Myeong-gi Jo}
\altaffiliation{
whaudrl4005@gmail.com
}
\author{Subin Kim}
\altaffiliation{
subini0213@snu.ac.kr
}
\author{Jeong Min Lee}
\altaffiliation{
jmleeluck@snu.ac.kr
}
\author{Eugene Park}
\altaffiliation{
eupark@snu.ac.kr
}
\affiliation{ 
Department of Physics and Astronomy, Seoul National University
}

\date{\today}
\begin{abstract}
this is abstract
\end{abstract}

\maketitle

\section{\label{sec:Intro} Introduction} 
this is main. 

\section{\label{sec:Method} Method}
An ADTL082J was used as the Op-Amp, and IRF1010E was used for the n-channel MOSFETs. The $I-V$ characteristics of the MOSFETs where measured prior to the experimental process, in order to obtain a gating voltage and the effective resisance when the MOSFET was open. The negative resistor was designed and measured with the circuit as in Fig. , and the current was measured via a resistor in series in the negative resistor($R$). The output voltage of the Op-Amp was also measured to compensate the saturation effects of the real world Op-Amp.

\section{\label{sec:Result} Result}
\subsection{Negative Resistance}
Fig 1
\subsection{Tuning Differential Resistance Slopes}
Fig 2
\subsection{Effect of Op-Amp Supply Voltage}
Fig 3
\subsection{Hysterisis Region}
Fig 4

\section{\label{sec:Conclusion} Conclusion}

\appendix

\section{\label{mosfetiv}MOSFET $I-V$ Characteristics}

\section{Derivation of the theoretical $I-V$ Characteristics of the Negative Resistor}
\subsubsection{\label{opampiv}Op-Amp}
For the 

\subsubsection{\label{opamp_mosfetiv}Op-Amp and MOSFETs}
The unsaturated region of the negativer resistor using enabling MOSFETs is identical to the unsaturated region of the negative resistor without the MOSFETs, since all of the MOSFETs do not flow current. We derive a theoretical description of the differential resistance slopes for the region where the MOSFETs are opened. 

\end{document}